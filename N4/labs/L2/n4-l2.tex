\documentclass{ip-lab}

\usepackage{url}
\usepackage{float}
\usepackage{hyperref}
\usepackage{upquote}

\labtype{Nivel 4}
\labnumber{Laboratorio 2}
\labtitle{Exploración de análisis de datos con Pandas y Matplotlib}

\makeheader

\begin{document}

\maketitle

\begin{sectionbox}{Objetivos}
\begin{enumerate}
    \item Explorar las funcionalidades básicas de la librería Pandas para análisis de datos.
    \item Comprender qué son las Series y DataFrames y cómo se utilizan.
    \item Investigar y aplicar métodos de conteo, agrupación y ordenamiento de datos.
    \item Familiarizarse con diferentes tipos de visualizaciones usando Pandas y Matplotlib.
    \item Desarrollar habilidades de documentación técnica y experimentación con librerías.
\end{enumerate}
\end{sectionbox}


\begin{sectionbox}{Preparación del ambiente de trabajo}  
  Este laboratorio es de \textbf{exploración}, lo que significa que usted investigará y experimentará con diferentes funcionalidades de estas librerías para comprender cómo funcionan. Deberá crear un archivo \texttt{exploracion\_datos.py} donde implementará sus experimentos.
  
  \textbf{Recursos de documentación oficial:}
  \begin{itemize}
      \item Pandas: \url{https://pandas.pydata.org/docs/}
      \item Matplotlib: \url{https://matplotlib.org/stable/contents.html}
  \end{itemize}
\end{sectionbox}

\pagebreak

\begin{sectionbox}{Introducción a Pandas}
  Pandas es una librería de Python especializada en el análisis y manipulación de datos. Se importa convencionalmente como \texttt{pd}:
  
  \begin{verbatim}
  import pandas as pd
  \end{verbatim}
  
  Pandas trabaja principalmente con dos estructuras de datos:
  
  \begin{itemize}
      \item \textbf{Series}: Una estructura unidimensional similar a una lista que puede contener diferentes tipos de datos (números, texto, etc.). Cada elemento tiene un índice asociado.
      
      \item \textbf{DataFrame}: Una estructura bidimensional similar a una tabla o matriz, donde los datos se organizan en filas y columnas. Cada columna es una Serie.
  \end{itemize}
  
  \textbf{Ejemplo de creación de una Serie:}
  \begin{verbatim}
  edades = pd.Series([23, 45, 18, 67, 34, 23, 45, 23])
  \end{verbatim}
  
  \textbf{Ejemplo de creación de un DataFrame:}
  \begin{verbatim}
  datos = {
      "nombre": ["Ana", "Luis", "María", "Pedro"],
      "edad": [23, 45, 18, 67],
      "ciudad": ["Bogotá", "Cali", "Bogotá", "Medellín"]
  }
  df = pd.DataFrame(datos)
  \end{verbatim}
\end{sectionbox}


\begin{sectionbox}{Actividad 1: Exploración de conteo en Series}
  \textbf{Investigación:} Pandas proporciona métodos para contar elementos en una Serie. Investigue en la documentación oficial qué hacen los siguientes métodos y cuál es la diferencia entre ellos:
  
  \begin{itemize}
      \item \texttt{pd.Series.value\_counts()}
      \item \texttt{pd.Series.count()}
  \end{itemize}
  
  \textbf{Experimentación:} Cree la siguiente Serie y aplique ambos métodos para observar sus diferencias:
  
  \begin{verbatim}
  calificaciones = pd.Series([4.5, 3.8, 4.5, 2.9, 4.5, 3.8, 5.0, 
                               4.5, None, 3.8, 4.5])
  \end{verbatim}
  
  \textbf{Preguntas para responder:}
  \begin{enumerate}
      \item ¿Qué información retorna \texttt{value\_counts()}? ¿En qué orden aparecen los resultados?
      \item ¿Qué tipo de valor retorna \texttt{count()}?
      \item ¿Cómo manejan estos métodos los valores \texttt{None} (valores nulos)?
      \item ¿En qué situaciones usaría cada uno de estos métodos?
  \end{enumerate}
\end{sectionbox}

\pagebreak

\begin{sectionbox}{Actividad 2: Exploración de operaciones con DataFrames}
  \textbf{Investigación:} Los DataFrames tienen métodos útiles para visualizar y ordenar datos. Investigue qué hacen los siguientes métodos:
  
  \begin{itemize}
      \item \texttt{pd.DataFrame.head(n)}
      \item \texttt{pd.DataFrame.tail(n)}
      \item \texttt{pd.DataFrame.sort\_values(ascending=flag)}
  \end{itemize}
  
  \textbf{Experimentación:} Cree el siguiente DataFrame y experimente con estos métodos:
  
  \begin{verbatim}
  estudiantes = pd.DataFrame({
      "nombre": ["Ana", "Luis", "María", "Pedro", "Sofía", 
                 "Carlos", "Laura", "Diego", "Valentina", "Andrés"],
      "edad": [20, 22, 19, 21, 20, 23, 19, 22, 20, 21],
      "nota": [4.5, 3.2, 4.8, 3.9, 4.1, 3.5, 4.7, 3.8, 4.2, 3.6]
  })
  \end{verbatim}
  
  \textbf{Preguntas para responder:}
  \begin{enumerate}
      \item ¿Qué filas muestra \texttt{head(3)} y qué filas muestra \texttt{tail(3)}?
      \item ¿Para qué sirve el parámetro \texttt{ascending} en \texttt{sort\_values()}?
      \item Ordene el DataFrame por la columna \texttt{nota}. ¿Qué pasa cuando \texttt{ascending=True}? ¿Y cuando \texttt{ascending=False}?
      \item ¿Cómo podría combinar estos métodos para ver solo los 5 estudiantes con mejores notas?
  \end{enumerate}
\end{sectionbox}

\pagebreak

\begin{sectionbox}{Actividad 3: Exploración de agrupación y cálculo de promedios}
  \textbf{Investigación:} Pandas permite agrupar datos y calcular estadísticas sobre esos grupos. Investigue qué hacen:
  
  \begin{itemize}
      \item \texttt{pd.DataFrame.groupby()}
      \item \texttt{pd.Series.mean()}
  \end{itemize}
  
  \textbf{Experimentación:} Cree el siguiente DataFrame con información de ventas:
  
  \begin{verbatim}
  ventas = pd.DataFrame({
      "producto": ["Laptop", "Mouse", "Laptop", "Teclado", "Mouse",
                   "Laptop", "Mouse", "Teclado", "Laptop", "Mouse"],
      "precio": [2500000, 45000, 2800000, 120000, 50000,
                 2600000, 48000, 125000, 2700000, 47000],
      "cantidad": [1, 3, 1, 2, 4, 1, 2, 1, 1, 5]
  })
  \end{verbatim}
  
  \textbf{Preguntas para responder:}
  \begin{enumerate}
      \item ¿Qué sucede cuando ejecuta \texttt{ventas.groupby("producto")}? ¿Qué tipo de resultado obtiene?
      \item Ejecute \texttt{ventas.groupby("producto")["precio"].mean()}. ¿Qué información le proporciona?
      \item ¿Cómo calcularía el promedio de cantidad vendida para cada producto?
      \item Experimente agrupando por producto y calculando el promedio de múltiples columnas a la vez.
  \end{enumerate}
\end{sectionbox}

\pagebreak

\begin{sectionbox}{Actividad 4: Exploración de gráficas de barras}
  \textbf{Investigación:} Pandas permite crear visualizaciones directamente desde Series y DataFrames usando el método \texttt{plot()}. Matplotlib es la librería que Pandas usa internamente para mostrar estas gráficas. Investigue:
  
  \begin{itemize}
      \item \texttt{pd.Series.plot(kind="bar")}
      \item \texttt{pd.Series.plot(kind="barh")}
      \item Los parámetros: \texttt{figsize}, \texttt{title}, \texttt{xlabel}, \texttt{ylabel}
      \item \texttt{plt.show()} de Matplotlib
  \end{itemize}
  
  Matplotlib se importa convencionalmente como:
  \begin{verbatim}
  import matplotlib.pyplot as plt
  \end{verbatim}
  
  \textbf{Experimentación:} Use los datos de la Actividad 3 y cree visualizaciones:
  
  \begin{verbatim}
  precios_promedio = ventas.groupby("producto")["precio"].mean()
  \end{verbatim}
  
  \textbf{Preguntas para responder:}
  \begin{enumerate}
      \item ¿Cuál es la diferencia visual entre \texttt{kind="bar"} y \texttt{kind="barh"}?
      \item ¿Qué hace el parámetro \texttt{figsize=(10, 6)}? Experimente con diferentes valores.
      \item ¿Para qué sirve \texttt{plt.show()}? ¿Qué pasa si no lo incluye?
      \item Cree una gráfica de barras horizontales con título \texttt{"Precio promedio por producto"}, etiqueta del eje X \texttt{"Precio (COP)"} y etiqueta del eje Y \texttt{"Producto"}.
      \item ¿En qué situaciones preferiría usar barras verticales vs. barras horizontales?
  \end{enumerate}
\end{sectionbox}

\pagebreak

\begin{sectionbox}{Actividad 5: Exploración de histogramas}
  \textbf{Investigación:} Los histogramas son útiles para visualizar la distribución de datos numéricos. Investigue:
  
  \begin{itemize}
      \item \texttt{pd.Series.plot(kind="hist")}
      \item ¿Qué diferencia hay entre un histograma y una gráfica de barras?
  \end{itemize}
  
  \textbf{Experimentación:} Cree una Serie con calificaciones de estudiantes:
  
  \begin{verbatim}
  import random
  random.seed(42)
  calificaciones = pd.Series([random.uniform(2.0, 5.0) 
                               for _ in range(100)])
  \end{verbatim}
  
  \textbf{Preguntas para responder:}
  \begin{enumerate}
      \item Cree un histograma de las calificaciones. ¿Qué información muestra?
      \item ¿Cómo se agrupan los datos en el eje X? ¿Qué representa el eje Y?
      \item Experimente con el parámetro \texttt{bins=20} dentro de \texttt{plot()}. ¿Qué cambia?
      \item ¿En qué situaciones es más útil un histograma que una gráfica de barras?
      \item Compare visualmente un histograma con 5 bins versus uno con 50 bins. ¿Cuál ofrece mejor información?
  \end{enumerate}
\end{sectionbox}

\pagebreak

\begin{sectionbox}{Actividad 6: Exploración de diagramas de caja (Boxplots)}
  \textbf{Investigación:} Los boxplots o diagramas de caja son útiles para visualizar la distribución de datos y detectar valores atípicos. Investigue:
  
  \begin{itemize}
      \item \texttt{pd.Series.plot(kind="box")}
      \item El parámetro \texttt{vert}
      \item ¿Qué información muestra un boxplot? (cuartiles, mediana, valores atípicos)
  \end{itemize}
  
  \textbf{Experimentación:} Use los datos de estudiantes de la Actividad 2:
  
  \begin{verbatim}
  estudiantes = pd.DataFrame({
      "nombre": ["Ana", "Luis", "María", "Pedro", "Sofía", 
                 "Carlos", "Laura", "Diego", "Valentina", "Andrés"],
      "edad": [20, 22, 19, 21, 20, 23, 19, 22, 20, 21],
      "nota": [4.5, 3.2, 4.8, 3.9, 4.1, 3.5, 4.7, 3.8, 4.2, 3.6]
  })
  \end{verbatim}
  
  \textbf{Preguntas para responder:}
  \begin{enumerate}
      \item Cree un boxplot de la columna \texttt{nota}. ¿Qué elementos visuales observa?
      \item ¿Qué representa la línea en el medio de la caja?
      \item ¿Cuál es la diferencia entre \texttt{vert=True} y \texttt{vert=False}?
      \item Investigue qué significan los "bigotes" (whiskers) y los puntos aislados en un boxplot.
      \item ¿Por qué sería útil un boxplot para comparar las notas de diferentes cursos?
  \end{enumerate}
\end{sectionbox}

\pagebreak

\begin{sectionbox}{Actividad 7: Exploración de manipulación de ejes}
  \textbf{Investigación:} Matplotlib permite manipular los ejes de las gráficas. Investigue:
  
  \begin{itemize}
      \item \texttt{ax.invert\_yaxis()} (donde \texttt{ax} es el objeto Axes de Matplotlib)
      \item ¿Cómo obtener el objeto \texttt{ax} al crear una gráfica con Pandas?
  \end{itemize}
  
  \textbf{Experimentación:} Use los precios promedio de la Actividad 4:
  
  \begin{verbatim}
  ventas = pd.DataFrame({
      "producto": ["Laptop", "Mouse", "Laptop", "Teclado", "Mouse",
                   "Laptop", "Mouse", "Teclado", "Laptop", "Mouse"],
      "precio": [2500000, 45000, 2800000, 120000, 50000,
                 2600000, 48000, 125000, 2700000, 47000]
  })
  
  precios_promedio = ventas.groupby("producto")["precio"].mean()
  precios_promedio = precios_promedio.sort_values()
  \end{verbatim}
  
  Para obtener el objeto \texttt{ax}, puede usar:
  \begin{verbatim}
  ax = precios_promedio.plot(kind="barh", figsize=(10, 6))
  \end{verbatim}
  
  \textbf{Preguntas para responder:}
  \begin{enumerate}
      \item Cree una gráfica de barras horizontales y luego aplique \texttt{ax.invert\_yaxis()}. ¿Qué cambia?
      \item ¿Para qué sería útil invertir el eje Y en una gráfica?
      \item Experimente invirtiendo el eje Y en diferentes tipos de gráficas (bar, barh). ¿En cuál tiene más sentido?
      \item Si ordena los datos de menor a mayor antes de graficar e invierte el eje Y, ¿cómo quedan visualmente ordenados los elementos?
  \end{enumerate}
\end{sectionbox}

\pagebreak

\begin{sectionbox}{Entrega}
Entregue un archivo comprimido (ZIP) que contenga:

\begin{enumerate}
    \item \textbf{exploracion\_datos.py}: Archivo con todo el código de sus experimentaciones, organizado por actividades. Use comentarios para separar cada actividad claramente.
    
    \item \textbf{respuestas.txt} o \textbf{respuestas.pdf}: Documento con las respuestas a todas las preguntas planteadas en las actividades. Incluya capturas de pantalla de las gráficas generadas cuando sea relevante.
\end{enumerate}

Suba el archivo comprimido a través de Brightspace en el laboratorio del Nivel 4 designado como "N4-L2: Exploración de análisis de datos con Pandas y Matplotlib".

\textbf{Recomendaciones:}
\begin{itemize}
    \item No se limite a responder las preguntas mínimas. Experimente con variaciones de los ejemplos.
    \item Consulte la documentación oficial cuando tenga dudas. Aprender a leer documentación es una habilidad fundamental.
    \item Pruebe combinaciones de métodos que no se mencionan explícitamente en el laboratorio.
    \item Si encuentra métodos o funcionalidades interesantes en la documentación, inclúyalos en su exploración.
    \item Comente su código para explicar qué está probando en cada sección.
\end{itemize}
\end{sectionbox}

\end{document}
