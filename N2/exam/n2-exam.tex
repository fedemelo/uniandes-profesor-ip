\documentclass{ip-exam}

\title{N2-EXAM -- 2025-20}
\examdate{26 de septiembre de 2025}

\begin{document}

\makeexamheader

\question{3}{¿Para qué sirve el método \texttt{get} de los diccionarios? ¿Cuándo es útil usarlo en lugar de la sintaxis convencional para leer el valor de una llave? Explique con un ejemplo.

\textit{Consejo: Sea conciso. No demore más de 5 minutos en este punto.}
}
\vspace{8cm}


\question{3}{Compare el método \texttt{find} con el método \texttt{count} de las cadenas de caracteres: ¿en qué se parecen y en qué difieren? Dé un ejemplo de cuándo usaría cada uno.

\textit{Consejo: Sea conciso. No demore más de 5 minutos en este punto.}
}
\vspace{3cm}

\newpage

\gridquestion[13cm]{13}{En un zoológico, tres animales compiten por ser el más destacado.

La información de cada animal se proporciona mediante un diccionario con cuatro llaves: \texttt{\textquotedbl especie\textquotedbl}, \texttt{\textquotedbl modo\_reproduccion\textquotedbl}, \texttt{\textquotedbl nombre\textquotedbl} y \texttt{\textquotedbl longitud\_patas\textquotedbl}.

E.g.:

\texttt{\{\\
\phantom{xxxx}\textquotedbl especie\textquotedbl: \textquotedbl jirafa\textquotedbl,\\
\phantom{xxxx}\textquotedbl modo\_reproduccion\textquotedbl: \textquotedbl vivíparo\textquotedbl,\\
\phantom{xxxx}\textquotedbl nombre\textquotedbl: \textquotedbl Juanita\textquotedbl,\\
\phantom{xxxx}\textquotedbl longitud\_patas\textquotedbl: 150\\
\}}

Implemente una función que recibe tres animales como parámetros y retorna la especie del animal con las patas más largas. En caso de empate en la longitud de las patas, dé prelación al animal que tenga el nombre propio más largo. En caso de empate total, prefiera el último diccionario en el orden de entrada.

\textit{Consejo: No demore más de 20 minutos en este punto.}}

\newpage

\gridquestion[13cm]{13}{En una registraduría arbitraria, cada uno de los nombres debe satisfacer las siguientes reglas:
\begin{itemize}
  \item Deben tener al menos una vez la letra \texttt{a}.
  \item Deben tener al menos 3 caracteres.
  \item No deben contener espacios en blanco.
\end{itemize}

Implemente una función que reciba tres nombres como parámetros y retorne un diccionario con los nombres como llaves, cada uno con un booleano como valor que indique si el nombre es válido o no.

\textit{Consejo: No demore más de 17 minutos en este punto.}}

\newpage

\gridquestion[11cm]{13}{Escriba una función que reciba tres números, cada uno de cuatro dígitos, y retorne un diccionario en donde cada uno de los tres números sea una llave y su valor sea:
\begin{itemize}
  \item La suma de sus dígitos, si la suma de los dígitos es par.
  \item La multiplicación de sus dígitos, si la suma de los dígitos es impar.
\end{itemize}

\textit{Consejo: No demore más de 17 minutos en este punto.}
}

\question{5}{Explique brevemente un concepto del nivel 2 del curso que no se evalúa en este examen.

\textit{Consejo: No demore más de 5 minutos en este punto.}
}

\vspace{3cm}

\end{document}
