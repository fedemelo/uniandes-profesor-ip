\documentclass{ip-lab}

\labtype{Nivel 2}
\labnumber{Laboratorio 2}
\labtitle{Condicionales}

\makeheader

\begin{document}

\maketitle

\begin{sectionbox}{Objetivos}
\begin{enumerate}
    \item Entender la utilidad de los condicionales y cómo aplicarlos a diferentes problemas.
    \item Practicar el uso de condicionales para la solución de problemas por casos.
\end{enumerate}
\end{sectionbox}


\begin{sectionbox}{Preparación del ambiente de trabajo}
  Cree un nuevo módulo de funciones llamado \texttt{condicionales.py} en el que escribirá las funciones correspondientes a los puntos 1 a 2 de este laboratorio.
\end{sectionbox}


\begin{sectionbox}{Actividad 1: Año bisiesto}
  Escriba una función que reciba como argumento un año y determine si el año es bisiesto o no.

  Un año es bisiesto si es divisible por 4, excepto si también es divisible por 100. En este último caso, solo será bisiesto si además es divisible por 400.
\end{sectionbox}

\begin{sectionbox}{Actividad 2: Validador de contraseñas}
Escriba una función que reciba como parámetro una contraseña y determine si cumple con los siguientes criterios de seguridad:

\begin{enumerate}
    \item La contraseña debe tener al menos 8 caracteres de longitud.
    \item Debe contener al menos una letra mayúscula.
    \item Debe contener al menos una letra minúscula.
    \item Debe contener al menos un dígito (0-9).

\pagebreak

    \item No debe contener espacios en blanco.
    \item No debe ser igual a las contraseñas comunes: \texttt{password}, \texttt{123456}, \texttt{admin}.
\end{enumerate}

La función debe retornar \texttt{True} si la contraseña cumple con todos los criterios, o \texttt{False} en caso contrario.
\end{sectionbox}

\begin{sectionbox}{Entrega}
Entregue el archivo \texttt{n2-l2.py} a través de Brightspace en el laboratorio del Nivel 2 designado como ``N2-L2: Condicionales''.
\end{sectionbox}

\end{document}
