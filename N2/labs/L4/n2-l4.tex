\documentclass{ip-lab}

\labtype{Nivel 2}
\labnumber{Laboratorio 4}
\labtitle{Diccionarios y Cadenas de Caracteres}

\makeheader

\begin{document}

\maketitle

\begin{sectionbox}{Objetivos}
\begin{enumerate}
    \item Practicar el uso de diccionarios para almacenar y procesar información estructurada.
    \item Aplicar métodos de cadenas de caracteres para análisis y validación de texto.
    \item Combinar condicionales con operaciones sobre diccionarios y strings.
\end{enumerate}
\end{sectionbox}

\begin{sectionbox}{Preparación del ambiente de trabajo}
  Cree un nuevo módulo de funciones llamado \texttt{n2\_l4.py} en el que escribirá las funciones correspondientes a las actividades 1 a 3 de este laboratorio.
\end{sectionbox}

\begin{sectionbox}{Actividad 1: Competencia de videojuegos}
En un torneo de videojuegos, cuatro jugadores compiten por ser el campeón. Cada jugador tiene información almacenada en un diccionario con las siguientes llaves: \texttt{\textquotedbl nickname\textquotedbl} (nombre de usuario), \texttt{\textquotedbl puntaje\textquotedbl} (puntaje obtenido), \texttt{\textquotedbl experiencia\textquotedbl} (años de experiencia), \texttt{\textquotedbl plataforma\textquotedbl} (consola utilizada), y \texttt{\textquotedbl region\textquotedbl} (región geográfica).

Implemente una función que reciba cuatro jugadores como parámetros y retorne el nickname del jugador con el puntaje más alto. En caso de empate en el puntaje, priorice al jugador que tenga más consonantes en su nickname y si persiste el empate, priorice al jugador que tenga menos experiencia. Se garantiza que no habrá empates completos.

\textbf{Pista:} Para contar consonantes, recuerde que las vocales en español son: a, e, i, o, u (tanto mayúsculas como minúsculas).
\end{sectionbox}

\pagebreak

\begin{sectionbox}{Actividad 2: Análisis de números palíndromos}
Escriba una función que reciba cuatro números de exactamente 6 dígitos y determine cuáles son palíndromos. Un número es palíndromo si se lee igual de izquierda a derecha que de derecha a izquierda.

La función debe retornar un diccionario donde cada número sea una llave y su valor sea un booleano indicando si es palíndromo.

Para los números \texttt{123321},\texttt{988789} \texttt{938726} y \texttt{927729}, la función debería retornar:

\texttt{\{\textquotedbl 123321\textquotedbl: True, \textquotedbl 988789\textquotedbl: False, \textquotedbl 938726\textquotedbl: False, \textquotedbl 927729\textquotedbl: True\}}

\textbf{Pista:} Separe los dígitos con aritmética o use indexación simple de cadenas de caracteres y compare.
\end{sectionbox}

\begin{sectionbox}{Actividad 3: Sistema de códigos de productos}
Una tienda en línea necesita validar códigos de productos que deben cumplir las siguientes reglas:
\begin{itemize}
  \item Deben tener exactamente 4 caracteres de longitud.
  \item No deben tener letras mayúsculas.
  \item No deben contener la secuencia \texttt{\textquotedbl 00\textquotedbl}.
\end{itemize}

Implemente una función que reciba cuatro códigos de productos como parámetros y retorne un diccionario donde cada código sea una llave y su valor sea un booleano indicando si el código es válido.

\textbf{Ejemplo:} Para los códigos \texttt{\textquotedbl a123\textquotedbl}, \texttt{\textquotedbl x00y\textquotedbl}, \texttt{\textquotedbl B456\textquotedbl}, \texttt{\textquotedbl m789\textquotedbl}, la función debería retornar:

\texttt{\{\textquotedbl a123\textquotedbl: True, \textquotedbl x00y\textquotedbl: False, \textquotedbl B456\textquotedbl: False, \textquotedbl m789\textquotedbl: True\}}
\end{sectionbox}

\begin{sectionbox}{Entrega}
Entregue el archivo \texttt{n2\_l4.py} a través de Brightspace en el laboratorio del Nivel 2 designado como ``N2-L4: Diccionarios y Cadenas''.
\end{sectionbox}

\end{document}
