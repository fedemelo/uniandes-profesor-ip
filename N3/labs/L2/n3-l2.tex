\documentclass{ip-lab}

\labtype{Nivel 3}
\labnumber{Laboratorio 2}
\labtitle{Ciclos sobre cadenas de caracteres}

\makeheader

\begin{document}

\maketitle

\begin{sectionbox}{Objetivos}
\begin{enumerate}
    \item Entender la utilidad de los ciclos \texttt{while} y cómo aplicarlos a diferentes problemas.
    \item Practicar el uso de ciclos para el procesamiento de cadenas de caracteres.
    \item Desarrollar habilidades para recorrer y manipular cadenas usando iteración.
\end{enumerate}
\end{sectionbox}


\begin{sectionbox}{Preparación del ambiente de trabajo}
  Para este laboratorio se le proporcionará un módulo de consola (\texttt{consola.py}) que contiene un menú interactivo para probar sus funciones.
  
  Usted debe crear un nuevo módulo de funciones llamado \texttt{modulo.py} en el que escribirá las funciones correspondientes a las actividades 1 a 3 de este laboratorio. La consola importará automáticamente sus funciones desde este módulo.
\end{sectionbox}


\begin{sectionbox}{Actividad 1: Invertir una cadena de caracteres}
  Escriba una función que reciba como parámetro una cadena de caracteres y retorne la cadena invertida.

  \textbf{Importante:} Debe implementar esta función usando un ciclo \texttt{while}. No puede usar la sintaxis de indexación negativa de Python (\texttt{cadena[::-1]}) ni métodos como \texttt{reversed()}. El objetivo es practicar el uso de ciclos para recorrer y construir cadenas carácter por carácter.

  \textbf{Ejemplo:} Si la cadena de entrada es \texttt{\textquotedbl Hola Mundo\textquotedbl}, la función debe retornar \texttt{\textquotedbl odnuM aloH\textquotedbl}.
\end{sectionbox}

\begin{sectionbox}{Actividad 2: Histograma de caracteres}
Escriba una función que reciba como parámetro una cadena de caracteres y retorne un diccionario que represente un histograma de los caracteres presentes en la cadena, es decir, cuántas veces aparece cada carácter en la cadena.

El diccionario debe tener como llaves cada carácter único que aparece en la cadena (incluyendo espacios, dígitos y símbolos), y como valores la cantidad de veces que ese carácter aparece.

\pagebreak

Debe implementar esta función usando un ciclo \texttt{while} para recorrer la cadena carácter por carácter.

\textbf{Ejemplo:} Si la cadena de entrada es \texttt{\textquotedbl hola hola\textquotedbl}, la función debe retornar:

\texttt{\{\textquotesingle h\textquotesingle: 2, \textquotesingle o\textquotesingle: 2, \textquotesingle l\textquotesingle: 2, \textquotesingle a\textquotesingle: 2, \textquotesingle\ \textquotesingle: 1\}}
\end{sectionbox}


\begin{sectionbox}{Actividad 3: Verificador de simetría de vocales}
Escriba una función que reciba como parámetro una cadena de caracteres y determine si las vocales en la cadena forman un patrón simétrico (es decir, si las vocales se leen igual de izquierda a derecha que de derecha a izquierda). Presuma que no hay tildes en la cadena.

La función debe:
\begin{itemize}
    \item Ignorar las consonantes, espacios y otros caracteres.
    \item Considerar solo las vocales (a, e, i, o, u), sin distinguir entre mayúsculas y minúsculas.
    \item Retornar \texttt{True} si las vocales forman un patrón simétrico, o \texttt{False} en caso contrario.
\end{itemize}

Debe implementar esta función usando ciclos \texttt{while} para extraer y comparar las vocales.

\textbf{Ejemplos:}
\begin{itemize}
    \item \texttt{\textquotedbl Anita lava la tina\textquotedbl} → \texttt{True} (vocales: a, i, a, a, a, a, i, a)
    \item \texttt{\textquotedbl Hola mundo\textquotedbl} → \texttt{False} (vocales: o, a, u, o)
    \item \texttt{\textquotedbl Oso\textquotedbl} → \texttt{True} (vocales: o, o)
\end{itemize}
\end{sectionbox}

\begin{sectionbox}{Entrega}
Entregue el archivo \texttt{modulo.py} a través de Brightspace en el laboratorio del Nivel 3 designado como ``N3-L2: Ciclos sobre cadenas de caracteres''.

Puede usar el módulo \texttt{consola.py} proporcionado para probar sus funciones antes de la entrega. Su archivo debe funcionar correctamente desde la consola.
\end{sectionbox}

\end{document}
