\documentclass{ip-lab}

\labtype{Nivel 1}
\labnumber{Laboratorio 2}
\labtitle{Funciones}

\makeheader

\begin{document}

\maketitle

\begin{sectionbox}{Objetivo general}
Entender cómo definir funciones que resuelven problemas y cuándo utilizarlas.
\end{sectionbox}

\begin{sectionbox}{Objetivos específicos}
\begin{enumerate}
    \item Trabajar con diferentes tipos de datos y conversión entre ellos.
    \item Practicar el uso de variables y la instrucción de asignación.
    \item Utilizar operaciones y funciones aritméticas en Python.
    \item Definir funciones que resuelven problemas.
    \item Utilizar funciones para resolver problemas.
\end{enumerate}
\end{sectionbox}

\begin{sectionbox}{Actividad 1: Velocidad del sonido}
Defina una función llamada \texttt{calcular\_velocidad\_sonido} que calcule la velocidad del sonido en el aire basándose en la temperatura. Use la fórmula
\begin{gather*}
    v = 331.3 + 0.606 \cdot T
\end{gather*}
donde $v$ es la velocidad en m s$^{-1}$ y $T$ es la temperatura en grados Celsius.
\end{sectionbox}

\begin{sectionbox}{Actividad 2: Tiempo de propagación del eco}
Escriba una función llamada \texttt{calcular\_tiempo\_sonido} que calcule cuánto tiempo tarda un eco en regresar. La función debe basarse en la temperatura ambiente y la distancia hasta el obstáculo que produce el eco (en metros). Recuerde que el sonido debe viajar ida y vuelta, por lo que la distancia total es el doble. 

Invoque la función \texttt{calcular\_tiempo\_sonido} para una temperatura de 20°C y una distancia de 100 metros. Muestre en pantalla el resultado en segundos.
\pagebreak

Consejos:
\begin{itemize}
    \item Reutilice la función \texttt{calcular\_velocidad\_sonido} del punto anterior.
    \item Utilice la fórmula: $\text{tiempo} = \dfrac{\text{distancia total}}{\text{velocidad}}$.
    \item Muestre el resultado en segundos usando \texttt{print}.
\end{itemize}
\end{sectionbox}

\begin{sectionbox}{Actividad 3: Sonido en Marte}
Modifique la función \texttt{calcular\_velocidad\_sonido} para que calcule la velocidad del sonido en la atmósfera de Marte. En Marte, la fórmula es diferente debido a la composición atmosférica:
\begin{gather*}
    v = 240 + 0.4 \cdot T
\end{gather*}
Redefina su función y ejecute nuevamente el programa de la Actividad 2 con los mismos valores de temperatura y distancia. Compare los resultados: ¿en qué planeta es más rápido el sonido?
\end{sectionbox}

\begin{sectionbox}{Entrega}
Cree un archivo comprimido .zip con el archivo \texttt{n2-l2.py}. Entregue el archivo comprimido a través de Brightspace en el laboratorio del Nivel 1 designado como ``L2: Funciones''.
\end{sectionbox}

\end{document}
