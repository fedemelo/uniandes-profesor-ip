\documentclass{ip-lab}

\labtype{Nivel 1}
\labnumber{Laboratorio 4}
\labtitle{Pre parcial}

\makeheader

\begin{document}

\maketitle

\begin{sectionbox}{Objetivo general}
Escribir un programa de juguete que resuelve un problema.
\end{sectionbox}

\begin{sectionbox}{Objetivos específicos}
\begin{enumerate}
    \item Escribir funciones de la capa de lógica, cuya responsabilidad es resolver un problema.
    \item Escribir funciones de la capa de presentación, cuya responsabilidad es solicitar la información necesaria al usuario y mostrar el resultado.
    \item Escribir un programa que invoque funciones de la capa de lógica y de la capa de presentación.
\end{enumerate}
\end{sectionbox}

\begin{sectionbox}{Actividad 1: Capa de lógica}
Escriba un módulo de Python con dos funciones:
\begin{enumerate}
    \item Una que calcule el cateto de un triángulo rectángulo, con base en el otro cateto y la hipotenusa. Puede usar el teorema de Pitágoras.
    \item Una función que reciba un número de horas transcurridas desde el 1 de enero a las 00:00 y determine en qué fecha se estaría (mes y día), suponiendo que cada mes tiene 30 días.
\end{enumerate}
\end{sectionbox}

\begin{sectionbox}{Actividad 2: Capa de presentación}
Escriba un módulo con al menos dos funciones:
\begin{enumerate}
    \item Una función que solicite al usuario el valor de un cateto de un triángulo rectángulo y el valor de la hipotenusa, y que con eso muestre el valor del otro cateto con un mensaje bonito.
    \item Una función que solicite al usuario un número de horas y muestre un mensaje indicando la fecha correspondiente (mes y día) calculada desde el 1 de enero.
\end{enumerate}
Puede escribir también un menú y una función de iniciar aplicación, \textbf{si lo desea}. 
\end{sectionbox}

\pagebreak

\begin{sectionbox}{Actividad 3: Pruebas}
Pruebe que el programa funciona desde la capa de presentación, es decir, desde la consola. Para eso:
\begin{itemize}
    \item Si escribió una función de iniciar aplicación, invóquela.
    \item Si no, invoque cada función de la capa de presentación directamente, una a la vez. Es decir: invoque primero una función, pruébela, luego puede comentar esa línea e invocar la siguiente función.
\end{itemize}

Puede usar los siguientes valores de prueba:
\begin{itemize}
  \item Para el triángulo rectángulo, $b = 3$ y $c = 5$. El valor del cateto faltante debería dar $4$.
  \item Para la conversión de horas a meses y días, $h = 1000$. La fecha debería representar el 11 de febrero (a las 16:00 horas, pero eso no lo pide el enunciado).
\end{itemize}

\end{sectionbox}

\begin{sectionbox}{Entrega}
  Cree un archivo comprimido .zip que contenga exactamente dos archivos .py: el módulo de la capa de lógica y el de la capa de presentación. Entregue el archivo comprimido a través de Brightspace en el laboratorio del Nivel 1 designado como ``L4: Pre parcial''.
\end{sectionbox}

\end{document}
