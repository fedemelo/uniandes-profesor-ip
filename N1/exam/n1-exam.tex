\documentclass{ip-exam}

\title{N1-EXAM -- 2025-20}
\examdate{29 de agosto de 2025}

\begin{document}

\makeexamheader

\question{3}{¿Es preferible programar en un IDE, como Spyder, o en una servilleta? Elija una posición y dé un par de razones.

\textit{Consejo: Sea conciso. No demore más de 5 minutos en este punto.}
}

\vspace{3cm}

\question{3}{¿Es útil usar funciones al programar? Dé un par de razones a favor o en contra.

\textit{Consejo: Sea conciso. No demore más de 5 minutos en este punto.}
}
\vspace{3cm}

\gridquestion{10}{Escriba una función en Python que calcule la estatura de una persona. La función debe recibir como parámetros el índice de masa corporal (IMC) y el peso en kilogramos. Debe retornar la estatura en metros. No utilice funciones de entrada o salida.

Recuerde que el índice de masa corporal se calcula como: $\text{IMC} = \dfrac{\text{peso}_\text{kg}}{(\text{estatura}_\text{m})^2}$

\textit{Consejo: No demore más de 10 minutos en este punto.}}

\gridquestion[10cm]{20}{Escriba una función que recibe como parámetro una cantidad de segundos. La función debe retornar la fecha y hora que resulta de sumar esos segundos a las 00:00:00 del 1 de enero de 1970, en el formato \texttt{año-mes-día hora:minuto:segundo}. E.g.,
\begin{itemize}
  \item Para un argumento de \texttt{0} segundos, la función debe retornar \texttt{"1970-1-1 0:0:0"}.
  \item Para un argumento de \texttt{65} segundos, la función debe retornar \texttt{"1970-1-1 0:1:5"}.
  \item Para un argumento de \texttt{1000000000} segundos, la función debe retornar \texttt{"2002-2-24 1:46:40"}.
\end{itemize}
Si quiere, suponga que todos los meses tienen 30 días.
Si no, sepa que el año 1972 fue bisiesto.

\textit{Consejo: No demore más de 30 minutos en este punto.}}

\newpage

\gridquestion[10cm]{9}{En las preguntas 3 y 4 escribió funciones de la capa de lógica. Elija su favorita. Escriba la función de la interfaz de usuario que corresponde a la función de lógica elegida. Es decir: escriba una función que solicite al usuario la información necesaria, la utilice para invocar la función de la capa de lógica, e imprima un mensaje descriptivo para el usuario.

\textit{Consejo: No demore más de 10 minutos en este punto.}
}

\question{5}{Explique brevemente un concepto del curso que no se evalúa en este examen.

\textit{Consejo: No demore más de 5 minutos en este punto.}
}

\vspace{3cm}

\end{document}
